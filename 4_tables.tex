The previous chapter considers a task on open-domain semi-structured
web pages (high \Breadth) that only requires a shallow
understanding of the input natural language query (low \Depth).
As the examples at the end of the previous chapter
show, compositional understanding of the query
and multi-step reasoning
are important for handling 
even seemingly simple questions.
%more interesting questions.

To simultaneously increase the scope of the knowledge source
(high \Breadth) and the complexity of the questions (high \Depth),
we consider a new task of answering complex questions
based on a given semi-structured web table.
As an illustration,
Figure~\ref{fig:wtq-setting} shows several
question-answer pairs and the associated table
from the \emph{Summer Olympic Games} article on Wikipedia.\footnote{The table was retrieved in 2015
and was heavily simplified for illustrative purposes.}
The questions have a wide range of complexity
and require various types of reasoning such as
comparison ($x_2$), superlatives ($x_3$), 
aggregation ($x_4$), and arithmetic ($x_5$).
Meanwhile, the table contains an arbitrary schema
depending on the whim of the article's editors,
so a system for solving the task needs to generalize
to potentially unseen data schema.

This chapter discusses the nature of the task
and compares it with related tasks and datasets.
Section~\ref{sec:wtq-task} formalizes the task,
while Section~\ref{sec:wtq-dataset} explains how we
collect a dataset, \wtq, with open-domain web tables
and complex questions.
Finally, in Section~\ref{sec:wtq-analysis},
we analyze the \Breadth and \Depth of the dataset
with respect to other related datasets.

\section{Task description}\label{sec:wtq-task}

In our task,
the system is given a context table $w$ (``world'')
and a question $x$ about the table.
The system should
output a list $y$ of values
that answers the question according to the table.
The task does not assume that the table follows
any particular format,
and the output values can be arbitrary strings.
Figure~\ref{fig:wtq-setting} shows an example
of a table $w$, a few question $x_i$, and their answers $y_i$.

At training time, the system has access to training data
$\{(x\i, w\i, y\i)\}_{i=1}^N$ of questions, tables, and answers.
In particular, the training data does not contain
\emph{how} the answers were derived from the questions.
To test that the system generalizes to unseen table schemata,
we ensure that the tables in training and test sets
do not overlap.

\begin{figure}[t]
\centering
\textsf{
\begin{tabular}{|l|l|l|l|} \hline
\textbf{Year} & \textbf{City} & \textbf{Country} & \textbf{Nations} \\ \hline
1896 & Athens & Greece & 14 \\
1900 & Paris & France & 24 \\
1904 & St.\ Louis & USA & 12 \\
\dots & \dots & \dots & \dots \\
2004 & Athens & Greece & 201 \\
2008 & Beijing & China & 204 \\
2012 & London & UK & 204 \\ \hline
\end{tabular}
}
\\[1em]
\newcommand{\myspace}{\rule{0pt}{1.2em}}
\begin{tabular}{@{\;}r@{: }l}
\myspace
$x_1$ & \emph{``Greece held its last Summer Olympics in which year?''} \\
$y_1$ & \{2004\} \\
\myspace
$x_2$ & \emph{``In which city's the first time with at least 20 nations?''} \\
$y_2$ & \{Paris\} \\
\myspace
$x_3$ & \emph{``Which years have the most participating countries?''} \\
$y_3$ & \{2008, 2012\} \\
\myspace
$x_4$ & \emph{``How many events were in Athens, Greece?''} \\
$y_4$ & \{2\} \\
\myspace
$x_5$ & \emph{``How many more participants were there in 1900 than in the first year?''} \\
$y_5$ & \{10\} \\
\end{tabular}
\caption{Our task is to answer a highly compositional question from an HTML table.
Each example in the training data contains a question $x$,
a table $w$,
and an answer $y$.}
\label{fig:wtq-setting}
\end{figure}

%\todo{Talk more about how the questions are complex}

\section{The \wtq dataset}\label{sec:wtq-dataset}

As a benchmark for the task,
we created a new dataset, \wtq, of question-answer pairs
on tables from Wikipedia.

The data collection process is as follows.
From the Wikipedia dump,
we randomly selected data tables with at least
8 rows and 5 columns.
We gave preference to tables with a high fraction of
numerical data, as more interesting questions
with comparison and calculation can be asked about the table.

\begin{table}[t]
\centering
\begin{tabular}{ll} \toprule
\textbf{Category} & \textbf{Prompts} \\ \midrule
Counting &
{How many \dots} \quad
{\dots number of \dots} \quad
must require counting \\
Superlatives &
{\dots the most \dots} \quad
{\dots the least \dots} \quad
{\dots top \dots} \\
& {\dots\,\blanko est (e.g., largest, tallest) \dots} \\
Comparison &
{\dots at least \dots} \quad
{\dots at most \dots} \quad
{\dots more/less than \dots} \\
& {\dots above/below \dots} \quad
{\dots same \dots as \dots} \quad
must require comparison \\
& {\dots\,\blanko er (e.g., larger, taller) \dots} \\
Arithmetic &
{How long \dots} \quad
{\dots difference \dots} \quad
{\dots average \dots} \\
& {\dots total \dots} \quad
must require calculation \\
Quantification &
{\dots no \dots} \quad
{\dots not \dots} \quad
{\dots each \dots} \quad
{\dots only \dots} \\
Conjunction &
{\dots and \dots} \quad
{\dots but \dots} \quad
{\dots or \dots} \quad
{\dots other \dots} \\
Anaphora &
{\dots his/her \dots} \quad
{\dots its/their \dots} \\
Ordering &
{\dots before \dots} \quad
{\dots after \dots} \quad
{\dots first \dots} \quad
{\dots last \dots} \\
& {\dots previous \dots} \quad
{\dots next \dots} \quad
{\dots consecutive \dots} \\ \bottomrule
\end{tabular}
\caption{Prompts for soliciting
questions with a diverse set of operations.}
\label{tab:turk-prompts}
\end{table}

After selecting the tables,
we created two tasks on Amazon Mechanical Turk,
a crowdsourcing service.
In the first task, we asked the workers to write four
trivia questions about the displayed table.
For each question,
we randomly gave the worker one of the 36 generic prompts
in Table~\ref{tab:turk-prompts}
to encourage more complex questions.
The worker can elect to change the prompt
if it is impossible or impractical to follow.

In the second task, we asked workers to answer
the questions from the first task on the given table.
To aid the workers, we provided shortcuts
for copying answers from the table,
counting the number of selected cells,
and computing statistics of the numbers in selected cells.
We only kept the answers that were agreed upon
by at least two workers.

The final dataset contains 22,033 examples
on 2,108 tables of various topics.
We set aside 20\% of the tables and their associated questions
as test set and only develop on the remaining examples.
We performed simple preprocessing on the tables:
we remove all non-textual contents
(e.g., images and hyperlink targets),
and if there is a merged cell spanning many rows or columns,
we unmerge it and copy the content into each unmerged cell.
In the later releases of the dataset,
we also cleaned up encoding issues and
enforced formatting consistencies among the answers.

\section{Dataset analysis}\label{sec:wtq-analysis}

With respect to contemporary datasets,
the \wtq dataset was designed to cover
a larger scope of the knowledge source (\Breadth)
and more complex questions (\Depth).
We will analyze the dataset along the following four aspects:
% as outlined in Section~\ref{sec:intro-motivation}:

\begin{itemize}
\item \emph{Diversity of data schema} (a \Breadth aspect).
The schema of an knowledge source defines the types of entities
and relations that can exist between entities.
For instance, in a single database table,
we can treat the field names
(e.g., $\{\emph{Id}, \emph{Name}, \dots\}$)
and field types (e.g., \T{VARCHAR})
as the schema for that table.
To increase \Breadth, we want the dataset to
contain knowledge sources with a large variety or an open set
of data schemata.

\item \emph{Knowledge coverage} (a \Breadth aspect).
Large knowledge bases
such as Freebase \cite{freebase2013dump}
contain much more knowledge than a
domain-specific database.
However, the amount of knowledge is bounded by
the fixed data schema
and the rate in which the knowledge base is populated.
An open-domain knowledge source,
such as the web,
covers a larger amount of up-to-date knowledge.

\item \emph{Compositionality} (a \Depth aspect).
One measure of task complexity
is the number of compositions or reasoning steps needed
to complete the task.
Note that due to sublexical compositionality
\cite{wang2015overnight},
the task can be compositional
even when the question itself is not.
For instance, finding someone's \emph{aunt} is compositional
if the knowledge source only allows querying
the parent, sibling, and gender of a person.

\item \emph{Types of operations} (a \Depth aspect).
Finally, we consider the diversity of operations
that are needed to compute the answers.
Other than the number of unique operations,
we also look at the properties of those operations
(e.g., what the operations take as inputs).

\end{itemize}

\subsection{Analysis of the \wtq dataset}

We now quantitatively analyze the \wtq dataset
along the different aspects of \Breadth and \Depth,
and compare the dataset with related datasets
when applicable.

\paragraph{Diversity of data schema (\Breadth).}

We consider the number of different fields
(indicated by the column header strings)
as a rough measure of data schema diversity.
The \wtq dataset contains 2,108 tables.
Among them, there are 13,396 columns
and 3,929 unique column headers.
The diversity of column headers
is much higher than traditional semantic parsing datasets,
such as \Sc{Geoquery} \cite{zelle96geoquery}
and \Sc{ATIS} \cite{dahl1994expanding},
which use small and fixed databases.

It is more tricky to compare our dataset
with the semantic parsing datasets on knowledge bases
such as \Sc{Free917} \cite{cai2013large}
and \Sc{WebQuestions} \cite{berant2013freebase}.
While large knowledge bases
can cover a large number of relations
among entities,
they usually have a predefined schema.
In contrast, tables from the web can have
any arbitrary schema depending on the authors
of the tables.

\paragraph{Knowledge coverage (\Breadth).}

We first compare the scope of knowledge
in our dataset with the information in knowledge bases.
% Knowledge bases are difficult to build
% and are usually incomplete.
To do so, we sample 50 examples from the \wtq dataset
and tried to answer them manually by
querying Freebase.
Even though Freebase contains some information
extracted from Wikipedia,
we can answer only 20\% of the questions,
indicating that the dataset covers
a broader range of knowledge than Freebase.

On the other hand,
it is also true that a large amount of knowledge
is not encoded as tables.
Nevertheless, we argue that parallel structures on the web page,
such as the product listing or the news feed,
can also be interpreted as a table
where each parallel structure becomes a table column.
For instance, in a product listing,
the product names can be detected based on the shared HTML properties,
and the identified elements form a ``column'' in our virtual table.
The same idea can be applied across different web pages
that were generated from the same template
(e.g., Amazon product pages).

Nevertheless,
regarding the distribution of questions,
one downside of our data collection mechanism
is that the crowd workers write the questions
\emph{after} seeing the tables.
This is the reverse of how an actual QA system would operate:
the user first asks the question,
and then relevant tables are retrieved
based on the question.
As such,
unlike the \Sc{OpenWeb} dataset in Chapter~\ref{chp:semi},
the distribution of questions is skewed away from
the natural distribution of questions.
Moreover,
questions that cannot be answered by the given table
are not present (except for annotation errors),
even though detecting unanswerable questions
is crucial for a real QA system.
We found these sacrifices to be necessary for our task setting:
most naturally occurring questions are either
not factual (\nl{How can I lose weight?}),
not complex (\nl{Where is Toronto?}),
or not readily suitable to be answered by a table,
and thus a considerable amount of data filtering effort
would be needed to construct a dataset with
a natural distribution of questions.

\paragraph{Compositionality (\Depth).}

One selling point of traditional semantic parsing datasets
is the complexity of the questions.
One extreme example from the \Sc{Geoquery} dataset
is the question
\nl{What states border states that border states that border states that border Texas?},
which requires four levels of reasoning.

While an average example in the \wtq dataset
is not as compositional,
a non-trivial fraction of examples
do involve at least three or four levels of reasoning.
We sample 200 examples from the training data
and note several questions that require compositional
reasoning in Table~\ref{tab:wtq-compositional-examples}.

\newcommand{\viewerLink}[1]{{\color{blue!50!black}\href{https://ppasupat.github.io/WikiTableQuestions/viewer/\##1}{#1}}}
\begin{table}[t]
\centering
\begin{tabular}{lp{5.3cm}p{7cm}} \toprule
\textbf{Link} & \textbf{Question} & \textbf{Comments} \\
\midrule
\viewerLink{203-705}
& how many people stayed at least 3 years in office?
& Requires computing the duration between
\emph{Took Office} and \emph{Leave Office} dates. \\
\viewerLink{203-116}
& which players played the same position as ardo kreek?
& Requires finding people with that \emph{Position},
but excluding Ardo Kreek himself. \\ 
\viewerLink{203-104}
& which athlete was from south korea after the year 2010?
& Requires filtering on two criteria,
one of which is a comparison. \\
\viewerLink{204-475}
& in how many games did the winning team score more than 4 points?
& Requires finding the winning team in each row
by comparing the two scores. \\
\viewerLink{204-920}
& how many consecutive friendly competitions did chalupny score in?
& Requires interpreting ``consecutive'',
which could be complex depending on the available operations
in the system \\
\viewerLink{204-725}
& who earned more medals--vietnam or indonesia?
& Requires comparing the \emph{Total} column
but only for the two countries in question. \\
\viewerLink{204-255}
& which district has the greatest total number of electorates?
& Requires summing up the \emph{Number of electorates}
in each district before comparing. \\
\viewerLink{204-157}
& how many games were only won by 20 points or less?
& Requires finding only the games with \emph{W} (win)
status, and with the difference between the scores
not exceeding 20, before counting them. \\
\viewerLink{203-109}
& what year did monaco ratify more international human rights treaties than they did in 1979?
& Requires counting the number of treaties ratified in each year,
and then find the year with more ratifications than 1979. \\
\viewerLink{203-146}
& jones won best actress in a play in 2005. which other award did she win that year?
& Requires finding the award name
(in the \emph{Category} column)
in 2005 but excluding
the ``Best Actress in a Play'' award. \\
\bottomrule
\end{tabular}
\caption[
Examples of compositional questions in the \wtq dataset.]{
Examples of compositional questions in the \wtq dataset.
The code in the \emph{Link} column is the table code
in the dataset.
}
\label{tab:wtq-compositional-examples}
\end{table}

In Section~\ref{sec:wtq-analysis-again},
we will return to quantitatively analyze
compositionality
by looking at the complexity of logical forms.

\paragraph{Types of operations (\Depth).}

The questions in the \wtq dataset require
a diverse set of operations to answer the questions.
We manually classified the 200 examples above
based on the types of operations
for answering the questions.
The statistics in Table~\ref{tab:wtq-operations}
shows that while a few questions only require
a simple operation such as table look-up or counting,
the majority of the questions
demands more complex operations.

\begin{table}[t]
\centering
\begin{tabular}{lr} \toprule
\textbf{Operation} & \textbf{Amount} \\ \midrule
look up cells in the specified column & 13.5\% \\
+ look at the next or previous rows & + 5.5\% \\
+ aggregation (e.g., counting, summing) & + 15.0\% \\
+ comparison (e.g., largest, highest) & + 24.5\% \\
+ arithmetic, union, intersection & + 20.5\% \\
+ other phenomena & + 21.0\% \\ \bottomrule
\end{tabular}
\caption[Operations required to answer questions in the \wtq dataset]{
The operations required to answer the questions
in 200 random examples from the \wtq dataset.}
\label{tab:wtq-operations}
\end{table}

One important aspect of our data collection procedure
is that we \emph{softly} specify the types of questions
via question prompts.
As Table~\ref{tab:wtq-operations} shows,
this results in a sizable number of questions
requiring the types of reasoning that we did not anticipate.
Some examples of such ``other phenomena''
include identifying the second-highest value
(\nl{what is the next highest hard drive available after the 30gb model?}),
summing up time periods
(\nl{how many people stayed at least 3 years in office?}),
or using world knowledge
(\nl{how far did they make it in the fa cup after 2009?}
requires knowing that \nl{Quarter-final} is better than \nl{Round of 16}).
This is in contrast to data collection methods
that restrict the types of questions more strongly.
As an extreme example,
the ``overnight'' data collection method
\cite{wang2015overnight}
generates a set of queries
(e.g., logical forms or database queries)
that the dataset wants to target,
translates them into natural language questions using templates,
and then asks humans to write natural paraphrases to the questions.
By design,
the resulting questions are guaranteed
to have corresponding queries,
but the diversity of questions is inherently limited
by the templates used to generate the queries.
Admittedly, we also observe a similar effect
in the \wtq dataset,
where the prompt words \nl{next} and \nl{previous}
cause the operation
``look at the next or previous rows'' to be slightly over-represented.

%%%%%%%%%%%%%%%%%%%%%%%%%%%%%%%%%%%%%

\subsection{Comparison with other semantic parsing for
question answering datasets}
\label{sec:wtq-other-datasets}

we provide an overview of semantic parsing datasets
in the literature for comparison.
For each dataset, we focus on its motivation
(i.e., what challenges it is trying to address) as well as 
the scope of the knowledge source (\Breadth)
and the complexity of the questions (\Depth).

\paragraph{Semantic parsing on databases.}
As explained in Section~\ref{sec:rw-knowledge-based-qa},
early semantic parsing datasets focus on answering
highly compositional questions (high \Depth)
on a small well-defined domain (low \Breadth).
Some of the representative datasets are described below:

\begin{itemize}
\item \Sc{Geoquery} \cite{zelle96geoquery} contains
880 geography questions that can be answered with a database
of 8 tables.
Each question is annotated with a Prolog program.
While the questions can be very compositional,
the database contains few relations (35 fields total)
and entities (737 unique non-numeric values).
A similar dataset, \Sc{Jobs} \cite{tang01ilp},
contains 640 questions about job postings.

\item \Sc{ATIS} \cite{dahl1994expanding}
is a popular dataset for 
natural language interfaces to database (NLIDB).
Previous work such as \citet{he2006spoken}
and \citet{zettlemoyer07relaxed}
uses 4,978 training and 448 test examples,
all of which were annotated with SQL queries.
The dataset also contains a database of 25 tables
with a total of 131 fields. % 24 links --> 107 unique fields.

\item \Sc{Overnight} \cite{wang2015overnight} datasets
was created using the ``overnight'' data collection framework:
generate logical forms,
translate them into questions using templates,
and then crowdsource natural paraphrases
of those questions to be used as the actual questions.
This results in 8 datasets from 8 different domains
that were collected within a day.
The datasets have recently been used as a
cross-domain semantic parsing dataset
\cite{su2017cross,herzig2018zeroshot}.

\end{itemize}

As mentioned previously, while the \wtq dataset does not include
extremely compositional (and arguably less natural) questions
like in \Sc{Geoquery} or related datasets,
it does contain examples with complex reasoning 
in the action space as demonstrated in Table~\ref{tab:wtq-compositional-examples}.

\paragraph{Semantic parsing on knowledge bases.}
To extend the scope of the knowledge source,
previous work has considered replacing small databases
with large knowledge bases,
which contain entities and relations
from a variety of domains.
The main focus of these dataset are mostly to scale
up to the new larger knowledge source (higher \Breadth),
and thus the question complexity is toned down
(lower \Depth).

\begin{itemize}
\item \Sc{Free917} \cite{cai2013large}
and
\Sc{WebQuestions} \cite{berant2013freebase}
contain factoid questions on Freebase,
a large knowledge base with more than 2 billion facts,
each describing a relation between two entities
(e.g., $(\T{BarackObama},\T{PlaceOfBirth},\T{Honolulu})$).
Some questions involve multi-step reasoning
(e.g., \nl{what is the name of justin bieber brother?}
requires filtering the answer by gender)
and operations such as counting
(e.g., \nl{how many australian states and territories?}).
However,
most questions are not compositional and
usually involve finding the correct anchor entity
(e.g., \T{BarackObama})
and a short sequence of relations to reach the answer
(e.g., \T{Spouse}--\T{PlaceOfBirth})
\cite{yao2014ie}.
The two datasets contain
917 and 5,810 questions, respectively.

While \Sc{Free917} is annotated with logical forms,
\Sc{WebQuestions} was designed to have only
a distant supervision
of question-answer pairs.
\citet{yih2016value} later cleans up
and annotates the questions with SPARQL queries,
resulting in the \Sc{WebQuestionsSP} dataset
of 4,737 answerable questions.

\item \Sc{SimpleQuestions} \cite{bordes2015simple}
is a large-scale dataset of more than 108,442 questions
on a subset of Freebase. The dataset was created to study
question answering at scale. As such, it exclusively contains
questions that only involve
identifying the correct anchor entity
and a single relation.

\item \Sc{SPADES} \cite{bisk2016evaluating}
contains 93,319 clozed-style questions
derived by blanking out an entity from
each declarative sentence.
The sentences themselves come from
a subset of \Sc{ClueWeb09}
that were annotated with Freebase entities \cite{gabrilovich2013facc1}.
While some sentences contain multiple entities,
the dataset as a whole is not very compositional.

\end{itemize}

To address the reduced \Depth of the datasets above,
more recent datasets on knowledge bases aim to increase
the complexity of the questions:

\begin{itemize}

\item \Sc{QALD} \cite{lopez2013evaluating}
is a series of shared tasks on question answering
over knowledge bases. Some of the most frequent challenges include
multilingual question answering, where the question can be from
any of the given list of languages;
and hybrid question answering, where both knowledge bases
and \emph{unstructured text} are needed to answer the questions.
While the scope of the knowledge source is large and
many questions are compositional,
especially in the hybrid task,
the training dataset is small (less than 500 examples),
and most participating systems resort to non-learning-based approaches.

\item \Sc{GraphQuestions} \cite{su2016graphquestions}
contains 5,166 questions
that can be answered by graph queries on Freebase.
In contrast to the other datasets above,
\Sc{GraphQuestions} contains
more compositional questions
with multiple relations, counting (e.g., \nl{how many \dots}),
comparison (e.g., \nl{\dots older than \dots}),
and superlatives (e.g., \nl{\dots most recent \dots}).

\item \Sc{ComplexWebQuestions} \cite{talmor2018web}
creates complex questions by modifying the SPARQL queries
from the \Sc{WebQuestionsSP} dataset.
The resulting dataset contains 34,689 questions
with complex operations such as conjunctions,
superlatives, comparatives, and compositions.

\end{itemize}

\paragraph{Semantic parsing on tables.}
After the release of the \wtq dataset,
several datasets of question answering on tables
have been released,
each of which targets a different aspect
of answering complex questions on tables.
Similar to the setup of \wtq,
the datasets below ensure that the tables in the training
and test sets do not overlap.

\begin{itemize}
\item \Sc{WikiSQL} \cite{zhong2017seq2sql}
has a similar setting to \wtq but focuses on generating SQL queries.
The dataset is large: it contains 80,654 SQL queries with
human-annotated descriptions on 24,241 Wikipedia tables.
All queries follow a similar pattern of selecting a column
(with an optional aggregation such as \T{COUNT})
and up to 4 \T{WHERE} clauses.
While this means the query pattern is not diverse,
multi-step reasoning is present in the form of multiple \T{WHERE} clauses.
Moreover, the dataset contains a large number of tables,
making it a suitable dataset
for learning to identify the correct table elements
based on the question in an open-domain setting.

\item \Sc{Spider} \cite{yu2018spider}
is a recent text-to-SQL dataset with a focus on high compositionality.
It contains 10,181 question-query pairs
on 200 databases, each with multiple tables.
In addition to the sheer number of SQL operations
and nested queries,
many examples also require using \T{JOIN} to connect the information
across tables.
Therefore, in addition to identifying the right SQL operators,
the system also needs to choose the table fields from a large set
of inter-related fields in the database.

\end{itemize}

\paragraph{Sequential semantic parsing.}
To extend beyond answering a single independent question,
previous work has proposed tasks of answering a series of
follow-up questions or performing actions sequentially.
The task complexity mainly manifests in the sequential nature
of the task.
To perform well in these tasks,
one needs to interpret the utterance under
the context of previous utterances
as well as the intermediate computation state.

\begin{itemize}
\item The original \Sc{ATIS} dataset \cite{dahl1994expanding}
contains long session data (7 turns on average)
where the user asks a sequence of follow-up questions
to the agent
(e.g., asking \nl{On American Airlines}
as a follow-up question to \nl{Show me flights
from Seattle to Boston next Monday}).
To perform well,
the system needs to keep track of the dialog context
and interprets the questions with respect to the query results
from the previous input utterances.
However, the majority of context-dependent utterances
only involve ellipsis:
the utterance adds, changes, or queries
some pieces of information
from the utterances preceding it
(e.g., the \nl{On American Airlines} example above
adds the airline to the query in the previous utterance).

\item \Sc{SQA} \cite{iyyer2017search}
contains sequential questions on Wikipedia tables.
The questions were written based on the compositional questions
in our \wtq dataset.
Concretely, each step of computation is converted into a question
annotated with the intermediate result (e.g., a set of selected cells).
Like most sequential semantic parsing datasets,
the questions feature context-dependent structures
such as anaphora (e.g., \nl{Which of \textbf{them} won a Tony after 1960?})
and ellipsis (e.g., \nl{Who had the biggest gap between their two award wins?} where the list of people considered is the filtered list
from the previous steps).

\item There are several datasets that involve
manipulating objects in a space
based on a sequence of utterances.
These datasets feature context-dependent spatial reasoning
as well as anaphora and ellipsis
(e.g., \nl{Then move to the left of it}).
Examples include
\Sc{SAIL} \cite{macmahon2006walk,chen11navigate},
where an agent navigates around a 2D grid map;
\Sc{SCONE} \cite{long2016projections},
where objects in a 1D space move, change properties,
or interact with other objects;
and \Sc{BLOCKS} \cite{bisk2016natural},
where blocks in 2D or 3D spaces are moved
based on either low-level or high-level commands.

\end{itemize}

It is arguable that using multiple context-dependent utterances
to describe a task is more natural than
forming a single complex question
\cite{iyyer2017search}.
However, this requires factoring the task into steps,
which might be difficult when the space of possible actions
and intermediate computation states
are not known by the user.
For instance,
during the creation of the \Sc{SQA} dataset above,
some questions in \wtq does not have clear intermediate computation results,
and the question writers were asked to write up a new question
in such cases.
Furthermore, some types of compositional computations can
be easily phrased as a single question
(e.g., \nl{which players played \textbf{the same} position
as ardo kreek?}).

\paragraph{Visual question answering.}
Images provide a nice playground for complex reasoning:
each object has various properties
(e.g., object type, color, size, amount)
and relations with other objects
(e.g., position, relative properties),
while the whole scene could describe actions
involving the objects.
Following the success of object recognition
\cite{krizhevsky2012imagenet,szegedy2015googlenet,he2016resnet}
and image captioning
\cite{farhadi2010every,lin2014microsoft,fang2015captions,mao2015deep},
researchers turn to the task of answering questions on images,
which requires more complex reasoning.

\begin{itemize}

\item Early visual QA datasets
focus on factoid questions.
For instance, \Sc{VQA} \cite{antol2015vqa}
contains free-form and open-ended questions on images.
Most images come from the MS COCO image captioning dataset
\cite{lin2014microsoft},
which contains a diverse set of photographs.
Answering the questions demands various skills
such as recognizing objects, recognizing activities,
and factual reasoning.
However, most questions require only one or two steps of reasoning,
and many could be guessed without looking at the image at all
\cite{agrawal2016analyzing,goyal2017making}.
A later dataset, \Sc{GQA} \cite{hudson2019gqa},
has a similar setting to \Sc{VQA}
but with more compositional questions.

\item \Sc{SHAPES} \cite{andreas2016neural}
and \Sc{CLEVR} \cite{johnson2017clevr}
address task complexity (\Depth) with highly compositional questions
on synthetic scenes.
The tasks are reminiscent of the questions in the SHRDLU system
\cite{winograd1972language} and feature a diverse types of reasoning
such as spatial reasoning, object property identification
(shape, size, color, and texture),
aggregation (e.g., \nl{How many \dots}),
and comparison (e.g., \nl{\dots same size as \dots}).
A follow-up dataset \Sc{CLRVR-Humans} \cite{johnson2017inferring}
provides human-authored sentences for the scenes in the \Sc{CLEVR} dataset.

\item In \Sc{NLVR} \cite{suhr2017nlvr} and \Sc{NLVR2} \cite{suhr2018nlvr2},
the task is to identify whether the statement describes the
image or not.
Similar to \Sc{CLEVR},
the questions are compositional,
and require both spatial and logical reasoning.
\Sc{NLVR} contains synthetic images,
while \Sc{NLVR2} contains real photographs.
 
\end{itemize}

\paragraph{Parsing into domain-specific languages.}

When the knowledge source does not expose its internal content,
a QA system needs to interact with it using an interface
specific to that knowledge source.
For instance, a web service might expose a set of APIs
that can be called,
and a QA system needs to predict the correct API function
and arguments.
% Most question answering and object manipulation tasks
% above can be approached using any logical form formalism
% or query language
% suitable for the task.
% On the other hand, we sometimes want to parse utterances
% into task-specific languages,
% which usually feature unique challenges.
We now give some examples of datasets whose outputs
are in specific programming languages
for interacting with such knowledge sources.
Note that some of these datasets are not QA datasets,
and their evaluation is based on whether the predicted program
matches the annotated program, either at the surface level
or at the behavioral level.

\begin{itemize}

\item \Sc{RegExp824} \cite{kushman2013regex}
and \Sc{NL-RX} \cite{locascio2016regex}
contain regular expressions and 
their natural language descriptions.
The system's output is evaluated on its equivalence with
the annotated regular expression.

One challenge highlighted by \citet{kushman2013regex}
is that while multiple
regular expressions are equivalent to each other,
some of them align better with the
natural language description.
A similar sentiment is shared by our dataset:
there are multiple ways to derive the answer.
However, while regular expression has a deterministic
mechanism for checking equivalence,
it is more difficult to distinguish correct ways
of deriving the answer from the ones that get
the right answer for wrong reasons.
We will study this phenomenon in detail in
Chapter~\ref{chp:dpd}.

\item \Sc{IFTTT} \cite{quirk2015language}
is a dataset of if-this-then-that recipes
annotated with short descriptions
(e.g., \nl{Archive your missed calls from Android to Google Drive}).
Each recipe contains a trigger and an action,
both of which are API functions
(e.g., \T{Google\_Drive}.\T{Add\_row\_to\_spreadsheet})
with several parameters.
While the number of distinct functions is limited,
the main challenge comes from the
noisy natural language descriptions,
which often under-specify the output.

\item \Sc{NL2Bash} \cite{lin2018nl2bash}
is a recent dataset of Bash one-liners.
While the number of Bash utilities is limited to 135,
each utility has a distinct parameter convention (i.e., distinct schema)
and should be considered as a separate domain.
Furthermore,
the commands contain a fair amount of compositionality
with logical connectives (\T{\&\&}, \T{||})
and nested commands (achieved via \T{|}, \T{\$(}\dots\T{)}, and \T{<(}\dots\T{)}).

% \end{itemize}

% \paragraph{Generating longer programs.}

% Recent works have considered generating functions or classes
% in general-purpose programming languages based on
% natural language descriptions
% and some additional contexts.
% The resulting programs contain
% multi-step computation
% with complex constructs such as branching and loops (high \Depth).

% \begin{itemize}

\item \Sc{Django} \cite{oda2015learning}
was originally a code-to-pseudocode dataset,
but has been reversed into a semantic parsing dataset \cite{ling2016latent}.
The output is a single line of Python from the Django project,
a real code base with a wide variety of constructs.

\item In the \Sc{MTG} and \Sc{Hearthstone} datasets \cite{ling2016latent},
the system is given the text and metadata of a card in a
collectible card game, and should output a class implementing the behavior
of the card.
While the task is in a closed domain,
the main challenge is to infer complex card behaviors
from a short high-level description.

\item In the \Sc{CONCODE} dataset \cite{iyer2018mapping},
the system is given a description of a Java method
along with the signatures of fields
and other methods in the surrounding class.
The system should then generate the method body.
Unlike \Sc{MTG} and \Sc{Hearthstone},
the \Sc{CONCODE} dataset operates in an open-domain setting:
the code and descriptions come from Github repositories,
and the output code has to be generated
based on what other methods are available in the class.

\item 
In the datasets above, the output is evaluated
on surface form metrics
(e.g., exact match and BLEU score)
instead of functional correctness.
Recent datasets such as
\Sc{AlgoLisp} \cite{polosukhin2018neural}
and \Sc{NAPS} \cite{zavershynskyi2018naps}
address this by providing test cases to verify the correctness of
the generated programs.
Similar to test-driven program synthesis tasks
\cite{gulwani2011automating,devlin2017robustfill,shi2019frangel},
a few test cases are also available to the system,
allowing it to search over the possible programs
to find one that pass the test cases.
The generated programs are then tested against hidden test cases.

\end{itemize}

In contrast to these datasets,
the \wtq dataset does not include logical forms or programs
for computing the answers.
Apart from saving the annotation cost,
the lack of logical forms creates an opportunity for
different types of approaches to solving the task,
including ones that do not use discrete symbols
to represent steps of reasoning.
And even among semantic parsing approaches,
the system is free to choose a logical form language
that suits the task or the model.
Nevertheless, as we will explore in the subsequent chapters,
having only the distant supervision from the annotated answers
makes it more difficult to learn a semantic parsing model.

\section{Conclusion}

We have presented a new question answering task
that features semi-structured tables from the web
as the knowledge source (high \Breadth)
and more complex questions that require
various types of reasoning and compositional understanding
(high \Depth).
The \wtq dataset we collected contains question-answer pairs
and not the steps for computing the answers,
making it a flexible dataset for different types of approaches.

In the next chapter,
we explore a semantic parsing approach for tackling the task.
To handle the unseen data schema,
the table will be converted to a generic graph representation.
The task is then to parse the question
into a domain-generic logical form
that can be executed on the graph to get an answer.
Using this framework,
we will highlight the challenges caused by the unseen schema
and question complexity,
and then address the challenges in subsequent chapters.
