\section{Summary}

\paragraph{New QA tasks with increased \Breadth and \Depth.}
In this dissertation, we focus on improving the capability
of question answering systems
%natural language interfaces, such as virtual assistants
%and natural language interface to databases,
along two axes.
First, to increase the scope of the knowledge sources
that the systems operate on
(\Breadth), we consider using web pages,
which contain open-domain semi-structured data,
as the knowledge source
(Chapter~\ref{chp:semi}).
As tabular data on web pages contain computable knowledge
suitable for complex reasoning,
we propose the task of answering compositional questions
on web tables,
which aims to increase task complexity (\Depth)
and the scope of the knowledge source simultaneously
(Chapter~\ref{chp:tables}).
For the task,
we have collected a new \wtq dataset of complex questions
on tables and their answers.
The dataset contains comparable or higher task complexity,
as well as a broader open-domain knowledge source,
than contemporary semantic parsing datasets.

\paragraph{Training a semantic parser from denotations.}
We cast the proposed QA task as learning
a semantic parser using the annotated answers as
distant supervision (Chapter~\ref{chp:parsing}).
We address the increased question complexity (\Depth)
by using compositional logical forms
to express the multiple steps needed for computing answers.
To construct candidate logical forms,
the parser uses a set of deduction rules
to build larger logical forms from smaller parts.

Meanwhile, to address the potentially unseen
table schema (\Breadth),
we represent tables with
schema-independent knowledge graphs,
and then use floating deduction rules to flexibly generate
open-domain logical form predicates
based on the table.
In contrast to offline information extraction,
which assumes a fixed and canonicalized database schema,
our on-the-fly information extraction approach
can adapt to arbitrary data schemata and
maintain uncertainty over the possible interpretations
of the table.
This ultimately allows us to answer a wider range
of open-domain questions.

\paragraph{Speeding up the parser.}
The coverage over answerable questions can be increased
by expanding the set of deduction rules
and augmenting the knowledge graph representation.
To compensate for the increased space of logical forms
from this expansion,
we proposed a way to improve search
by reusing useful logical form patterns
(Chapter~\ref{chp:macro}).
When processing an input question,
similar questions in the training dataset are retrieved,
and the logical form macros from those questions
are used to construct logical forms.
This allows the parser to skip the generation of
logical forms that are less likely to be a good parse.

\paragraph{Converting distant supervision to direct supervision.}
Training a parser with distant supervision requires
performing search during training,
which introduces several challenges.
With the expanding space of logical forms,
it is more difficult to find logical forms consistent
with the correct answer during training,
and the discovered logical forms might be spurious
(right for wrong reasons),
leading to bad parameter updates.
We propose to factor out the search
and precompute the semantically correct logical forms
for each example,
which effectively converts the dataset into
a directly supervised one (Chapter~\ref{chp:dpd}).
The resulting logical forms
can then be used to train a parser
that takes logical forms as supervision
without having to perform search during training.

\section{State-of-the-art on the \wtq dataset}

\begin{table}[t]
\centering
\begin{tabular}{lrr} \toprule
& \multicolumn{2}{c}{\textbf{Test accuracy}} \\
\cmidrule{2-3}
\textbf{System}
& \textbf{Single model}
& \textbf{Ensemble} \\ \midrule
Original deduction rules (Chapter~\ref{chp:parsing})
& 37.1
& - \\
Expanded deduction rules (Chapter~\ref{chp:macro})
& 42.7
& - \\
Expanded deduction rules + macros (Chapter~\ref{chp:macro})
& 43.7
& - \\ 
\midrule

Neural Programmer \cite{neelakantan2017learning}
& 34.2
& 37.7 \\
Neural Multi-Step Reasoning \cite{haug2018neural}
& 34.8
& 38.7 \\
Rule-based + Abductive Matching \cite{dhamdhere2017abductive}
& 40.4
& - \\
Neural Decoding with Type Constraints \cite{krishnamurthy2017neural}
& 43.3
& 45.9 \\
Memory Augmented Policy Optimization (MAPO) \cite{liang2018mapo}
& 43.9
& 46.3 \\
MAPO + Meta Reward Learning \cite{agarwal2019learning}
& 44.1
& 46.9 \\
MAPO + Neural Program Planner \cite{biloki2019neural}
& 43.9
& 47.2 \\
\bottomrule

\end{tabular}

\caption{Test accuracy of the previous work
on the \wtq dataset.}
\label{tab:conclude-sota}
\end{table}

Table~\ref{tab:conclude-sota}
summarizes previous work on the \wtq dataset.
Descriptions and insights of the methods and their related works
are described below.
The annotated answers are used as distant supervision
unless stated otherwise.

\subsection{Using continuous representations for computation}

Logical form operators are interpretable
and efficient to execute.
However,
a system generating such symbolic operators as discrete actions
cannot be trained end-to-end with backpropagation.
As an alternative,
the following works consider using 
differentiable operators
to represent the steps of computing the answer.

\begin{itemize}
\item 
Neural Programmer \cite{neelakantan2016neural,neelakantan2017learning}
proposes to model the operators as predefined continuous functions
such as matrix multiplication and pooling.
At each time step, the model
applies each operator on the current state vectors,
and then compute a weighted average of the results
(where the weight of each operator is predicted by the model)
to be used as new state vectors.
The final state vectors are used to compute the final answer.

Since all operators are differentiable,
the model can be trained end-to-end with backpropagation
from distant supervision.
The model will have to learn to compute the weights of the operators
based on the input utterance and table.

\item
Similar to Neural Programmer,
Neural Enquirer \cite{yin2016neural}
also uses continuous functions to model the steps of computation.
However,
the hand-engineered networks for the operators in Neural Programmer
are replaced with a single general network
for computing the state vectors.
The network can be trained using a loss function
defined on the final denotation.
The system is evaluated on synthetic questions that are compositional
but closed-domain.

\item
The work by \citet{mou2017coupling}
models the steps of computations in two ways:
continuous functions,
which are trained end-to-end;
and symbolic operators,
whose decoder is trained with REINFORCE \cite{williams1992simple}.
The key observation is that parts of the state vectors
from the continuous functions are interpretable
(e.g., as attention over table rows).
Such information
can serve as a guidance when learning to decode symbolic operators.
This helps mitigate the cold start problem
of REINFORCE by boosting the chance of getting reward
when training the symbolic decoder.
The system is evaluated on the dataset by \citet{yin2016neural} above.

\end{itemize}

Continuous operators allows the model to be trained
in an end-to-end fashion from denotations without having to
perform search.
However, it can be difficult to design or learn new operators
to perform more complex reasoning;
for instance, adding an operator to subtract one column from another
would require either a careful design for a predefined operator,
or a large amount of training data for a trainable operator.
Trainable operators like in \citet{yin2016neural}
and \citet{mou2017coupling} are also less interpretable
than the discrete logical forms.

\subsection{Different generation and scoring mechanisms}

The following works use logical forms to represent
the steps of computation,
but employ different methods for generating
and scoring the logical forms.

\begin{itemize}

\item Neural Multi-Step Reasoning \cite{haug2018neural}
uses the floating parser (Chapter~\ref{chp:parsing})
to generate logical form candidates,
but then applies a neural scorer to score logical forms.
In particular, the scorer uses
pre-trained word embeddings and information from the table to
embed the utterance
and a natural language paraphrase of the logical form
into the same space.
The similarity score between the two embeddings becomes the score
of the logical form.

\item The work by \citet{dhamdhere2017analyza,dhamdhere2017abductive}
uses a rule-based system to
process the table,
generate semantic parses from the input,
and score them.
To fill in the operands that the rule-based system cannot handle well
(e.g., matching the phrase \nl{movie} to the column \emph{Title}),
a machine learning system is trained to perform abductive reasoning
and infer the missing values.
The rule-based system is easy to debug
and generates a controllable number of parses
(8.7 on average as opposed to exponentially many parses from the
floating parser),
while the machine learning part helps with
the difficult and open-domain predicates.

\item Neural Decoding with Type Constraints \cite{krishnamurthy2017neural}
trains a top-down symbolic decoder using the logical forms
generated by DPD and fictitious tables (Chapter~\ref{chp:dpd})
as supervision.
Type constraints mined from logical forms
are used to ensure the validity of the output,
as well as to control the space of possible actions at each time step.

\item Memory Augmented Policy Optimization (MAPO) \cite{liang2018mapo}
applies a policy gradient algorithm to train an agent
that generates logical forms in Lisp-like syntax.
To address the cold start problem when training from a random policy,
the work introduces (1)
systematic exploration that caches the explored trajectories
so that consistent logical forms are found more quickly;
(2) a memory buffer
for storing logical forms with high reward;
and (3) an unbiased low-variance policy gradient update
that incorporates both logical forms in the memory buffer
and on-policy trajectories.
Later works improves the reinforcement learning part
with meta reward learning \cite{agarwal2019learning}
and a model-based program planner \cite{biloki2019neural}.

% \item The Sequential Question Answering dataset \cite{iyyer2017search} \todo{}

% \item \citet{misra2018policy} \todo{}

\end{itemize}

\section{Future directions}

\paragraph{Knowing when to abstain.}
Our main evaluation metric, accuracy,
encourages the model to always predict an answer
even when (1) the question cannot be answered from the given context,
and (2) the model is not confident about the answer.
This is problematic is real-world settings
where precision is more important than recall;
for instance, a virtual assistant should not perform actions
with misleading or disastrous consequences
\cite{dhamdhere2017abductive}.
One possible future direction is to make a model
that can abstain from answering a question,
and can make a trade-off between precision and recall
\cite{yao2014freebase,rajpurkar2018squadrun,dong2018confidence}.

\paragraph{Avoiding the long tail of phenomena.}
\wtq is a challenging dataset containing both compositional utterances
with little constraints, and tables with arbitrary data and formatting.
As such, while the model errors can be categorized at a high-level
(Section~\ref{sec:sempre-error-analysis}),
addressing an individual error type with some heavy machinery
would push up the score by only a small amount.
Moreover, interesting phenomena
that occur less frequently in the dataset
will likely to be ignored by the model.

One alternative solution is to come up with a family of datasets,
each focusing on some particular challenge (e.g., rare logical operators
or table understanding) while keeping the rest fixed.
This way, methods for handling the phenomena could be developed
more quickly with clear results.
Another potential solution is to gather new examples
for the rare phenomena in an interactive fashion.

\paragraph{Combining information from multiple sources.}
Finally, the tasks introduced in this dissertation
are limited to using one source of data for all computation steps.
In reality, the information needed to perform a task
is often scattered around in multiple knowledge sources.
For instance, to handle the query \nl{Direction to John's house},
the system would need to identify who John is
(e.g., by finding a close friend on Facebook with name John),
find his address
(e.g., by looking up on John's personal web page),
and then find the direction
(e.g., using GPS and a map application).

Several question answering and semantic parsing datasets
such as QALD \cite{lopez2013evaluating},
Spider \cite{yu2018spider},
HotpotQA \cite{yang2018hotpotqa},
and DROP \cite{dua2019drop}
require the fusion of knowledge from multiple sources.
We believe that the ability to plan complex actions
on multiple, open-domain environments is the right path
toward an ideal natural language interface.
