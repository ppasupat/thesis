\chapter{Introduction}\label{chp:intro}

\section{Motivation}

\subsection{Academic motivation}
We want to work toward true natural language understanding.

Previous work on question answering
and information extraction tend to focus on one of:

\begin{itemize}
\item \textbf{Breadth.}
The size of the domain or knowledge source.
The ideal case is open-domain (e.g., retrieval + open QA).
\item \textbf{Depth.}
The complexity of the process required to get the output.
The ideal case is when the process is a composition of
multiple steps (e.g., database query).
\end{itemize}

We propose a framework that handle 
\begin{itemize}
\item \textbf{Breadth.}
We choose web pages.
In particular, at test time, we test on unseen contexts.
\item \textbf{Depth.}
We want to handle various operations
(comparison, computation, aggregation, spatial reasoning, etc.).
To allow this to happen, we slightly restrict the breadth
and look at semi-structured data such as
web page layouts, lists, and tables
(i.e., not the paragraph text).
\end{itemize}

\subsection{Practical motivation}

%%%%%%%%%%%%%%%%%%%%%%%%%%%%%%

\section{Thesis outline}

\section{Summary of contributions}